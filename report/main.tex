\newif\ifAnon\Anonfalse
\newif\ifDraft\Draftfalse

\documentclass[letterpaper,twocolumn,10pt]{article}
\usepackage{tugraz_defaults}
\usepackage{usenix}
\microtypecontext{spacing=nonfrench}
\usepackage{hyphenat}
\usepackage{xcolor,colortbl}
\usepackage{enumitem}
\usepackage{subcaption}
\usepackage[most]{tcolorbox}
\tcbset{frame style={color=black!30!white}, coltitle=black, boxsep=1pt, boxsep=0pt,left=3pt,right=3pt,top=3pt,bottom=3pt}
\usepackage{titlesec}
\usepackage{datatool}
\DTLsetseparator{ = }

\DTLloaddb[noheader, keys={mykey,myvalue}]{covert}{results_dat/covert_results.dat}
\DTLloaddb[noheader, keys={mykey,myvalue}]{keystroke}{results_dat/keystroke_attack_result.dat}
\DTLloaddb[noheader, keys={mykey,myvalue}]{noteboard}{results_dat/noteboard-result.dat}
\DTLloaddb[noheader, keys={mykey,myvalue}]{remote-covert}{results_dat/remote-covert-result.dat}

%Use this like this: /accessdatfile{covert}{ait_acc_mean}
\newcommand{\accessdatfile}[2]{\DTLfetch{#1}{mykey}{#2}{myvalue}}


\newcommand{\missingcommand}[1]{\DTLfetch{mydata}{thekey}{#1}{thevalue}}
\begin{document}

% paper title
\title{Side-Channel Attacks on Optane Persistent Memory: Artefact Evaluation Report}

% make the title area
\maketitle
% \thispagestyle{firstpage}

\newcommand{\corresponds}[1]{\textbf{\\Corresponds to #1  in our original paper.}}

\section{Reverse Engineering}
\subsection{General Heirarchy}
The following figure shows how Optane average read latency varies with increasing working set sizes.
\corresponds{figure 3}
\begin{figure}[H]
\centering
\adjustbox{max width=\hsize}{\begin{tikzpicture}
\begin{axis}[
mlineplot,
style={font=\footnotesize},
xlabel={Memory footprint (Bytes)},
ylabel={Read latency (ns)},
% ylabel style={text width=2.5cm,align=center},
width=1\hsize,
scaled y ticks=false,
xtick pos=bottom,
ytick pos=left,
xmin=64,
xmax=67108864,
xtick={64,256,1024,4096,16384,65536,262144,1048576,4194304,16777216,67108864},
xticklabels={64,256,{1kB},{\SI{4}{\kilo\byte}},{16kB},{6\SI{4}{\kilo\byte}},{256kB},{1MB},{4MB},{\SI{16}{\mega\byte}},{64MB}},
ymin=150,
ymax=450,
xmode=log,
height=3.5cm,
yticklabel style={/pgf/number format/1000 sep=},
xticklabel style={rotate=45,/pgf/number format/1000 sep=},
]
\addplot+[thick,mark=none,y filter/.expression={y/2.1}] table[x=region_size,y=average_latency,col sep=comma] {results_csv/gen_hier.csv};
\draw[gray,densely dotted,thick] (axis cs:16384,150) -- (axis cs:16384,500);
\node[draw=none] at (axis cs:800,270) {RMW buffer size};
\draw[->,black,thick] (axis cs:4200,240) -- (axis cs:16384,210);
\draw[gray,densely dotted,thick] (axis cs:16777216,150) -- (axis cs:16777216,500);
\node[draw=none] at (axis cs:1400000,420) {AIT buffer size};
\draw[->,black,thick] (axis cs:8000000,390) -- (axis cs:16000000,350);
\end{axis}
\end{tikzpicture}
}
\label{fig:general_hierarchy}
\vspace{-0.2cm}
\end{figure}

\subsection{RMW Results}
The following figures depict the RMW buffer's associativity and replacement policy.
\corresponds{figure 5}
\begin{figure}[H]
\centering
\begin{subfigure}[b]{0.49\hsize}
\begin{tikzpicture}
\begin{axis}[
mlineplot,
style={font=\footnotesize},
xlabel={Memory footprint (Bytes)},
ylabel={Read latency (ns)},
% title={(a) Same order},
ylabel style={text width=3cm,align=center},
width=1\hsize,
scaled y ticks=false,
xtick pos=bottom,
ytick pos=left,
xmin=256,
xmax=65536,
xtick={256,1024,4096,16384,65536},
xticklabels={256, 1kB, \SI{4}{\kilo\byte}, 16kB, 6\SI{4}{\kilo\byte}},
ymin=100,
ymax=400,
xmode=log,
height=3.5cm,
legend columns=2,
legend cell align=left,
legend image post style={scale=0.6}, 
legend style={nodes={scale=0.75},draw=none,anchor=north,at={(0.5,1.6)}},
]
\addplot+[thick,mark=*,mark size=0.5,y filter/.expression={y/2.1}] table[x=size,y=mask_clean,col sep=comma] {results_csv/rmw_assoc.csv};
\addplot+[thick,mark=*,mark size=0.5,y filter/.expression={y/2.1}] table[x=size,y=mask_8,col sep=comma] {results_csv/rmw_assoc.csv};
\addplot+[thick,mark=*,mark size=0.5,y filter/.expression={y/2.1}] table[x=size,y=mask_9,col sep=comma] {results_csv/rmw_assoc.csv};
\addplot+[thick,mark=*,mark size=0.5,y filter/.expression={y/2.1}] table[x=size,y=mask_10,col sep=comma] {results_csv/rmw_assoc.csv};
\addplot+[thick,mark=*,mark size=0.5,y filter/.expression={y/2.1}] table[x=size,y=mask_20,col sep=comma] {results_csv/rmw_assoc.csv};
\addplot+[thick,mark=*,mark size=0.5,y filter/.expression={y/2.1}] table[x=size,y=mask_21,col sep=comma] {results_csv/rmw_assoc.csv};
\node[draw=none, text width=2.3cm] at (axis cs:4096,330) {Results of different bitmasks overlap};
\draw[->,black,thick] (axis cs:4096,290) -- (axis cs:16384,250);
\legend{No mask, mask bit 8, mask bit 9, mask bit 10, mask bit 20, mask bit 21}
\end{axis}
\end{tikzpicture}
\vspace{-0.5cm}

\end{subfigure}
\begin{subfigure}[b]{0.49\hsize}
\vspace{-0.4cm}
\begin{tikzpicture}
\begin{axis}[
mlineplot,
style={font=\footnotesize},
xlabel={\# RMW cache lines ($N$)},
ylabel={RMW miss rate},
% title={(b) Reverse order},
ylabel style={text width=3cm,align=center},
width=1\hsize,
scaled y ticks=false,
xtick pos=bottom,
ytick pos=left,
xmin=1,
xmax=128,
xtick={1,32,...,128},
xticklabels={1,32,64,96,128},
ymin=0,
ymax=1,
height=3.5cm,
legend columns=1,
legend cell align=left,
legend image post style={scale=0.6}, 
legend style={nodes={scale=0.75},draw=none,anchor=north,at={(0.5,1.6)}},
]
\addplot+[thick,mark=triangle,mark size=1,draw=blue,mark indices={4,8,16,...,128}] table[x=Iteration,y=Rate,col sep=comma] {results_csv/rmw_replacement_same_order.csv};
\addplot+[thick,mark=*,mark size=1,draw=red!60,mark indices={4,8,16,...,128}] table[x=Iteration,y=Rate,col sep=comma] {results_csv/rmw_replacement_reverse_order.csv};
\addplot+[thick,mark=o,mark size=1,draw=black!20,mark indices={4,8,16,...,128}] table[x=Iteration,y=Rate,col sep=comma] {results_csv/rmw_replacement_random_order.csv};
\legend{Same order, Reverse order, Random order}
\end{axis}
\end{tikzpicture}

\end{subfigure}
\label{fig:rmw_reverse}
\end{figure}

%\begin{figure}[t]
%% \vspace{-0.8cm}
%\centering
%\begin{tikzpicture}
\begin{axis}[
ybar,
style={font=\footnotesize},
xlabel={Read latency (ns)},
ylabel={Frequency [\%]},
% ylabel={Density},
ylabel style={text width=2.5cm,align=center},
width=1\hsize,
scaled y ticks=false,
xtick pos=bottom,
ytick pos=left,
ytick = {0,0.2,0.4},
yticklabels={0\,\%,20\,\%,40\,\%},
xmin=100,
xmax=450,
ymin=0,
ymax=0.5,
height=3.3cm,
legend columns=2,
legend cell align=left,
legend image post style={scale=0.6}, 
legend style={nodes={scale=0.75},draw=none,anchor=north,at={(0.5,1.3)}},
]
\addplot+[
    hist={bins=200,data min=0, data max=1134},
    y filter/.expression={y==0 ? -10 : y/100},
    x filter/.expression={x/2.1},
] table[y index=0] {results_csv/read_wo_clflush.csv};
\addplot+[
    hist={bins=200,data min=0, data max=1134},
    y filter/.expression={y==0 ? -10 : y/100},
    x filter/.expression={x/2.1},
    pattern=north east lines,
] table[y index=0] {results_csv/read_w_clflush.csv};
\legend{{Normal RMW hit latency~~}, {With \texttt{CLFLUSH}}}
\end{axis}
\end{tikzpicture}
\vspace{-0.4cm}

%\caption{Effect of \texttt{CLFLUSH} to RMW buffer.}
%\label{fig:rmw_flush}
%\vspace{-0.4cm}
%\end{figure}

\subsection{AIT Results}
The following figures depict the AIT buffer's associativity and replacement policy.
\corresponds{figure 7}
\begin{figure}[H]
\centering
\begin{subfigure}[b]{0.52\hsize}
\adjustbox{}{\begin{tikzpicture}
\begin{axis}[
mlineplot,
style={font=\footnotesize},
xlabel={Memory footprint (Bytes)},
ylabel={Read latency (ns)},
% title={(a) Same order},
% ylabel style={text width=2cm,align=center},
width=1\hsize,
scaled y ticks=false,
xtick pos=bottom,
ytick pos=left,
xmin=8192,
xmax=33554432,
xtick={4096,65536,1048576,16777216},
xticklabels={\SI{4}{\kilo\byte},6\SI{4}{\kilo\byte},1MB,\SI{16}{\mega\byte}},
ymin=100,
ymax=500,
xmode=log,
height=3.5cm,
legend columns=2,
legend cell align=left,
legend image post style={scale=0.6}, 
legend style={nodes={scale=0.75},draw=none,anchor=north,at={(0.5,1.6)}},
]
\addplot+[thick,mark=none,y filter/.expression={y/2.1},black] table[x=size,y=mask_clean,col sep=comma] {results_csv/ait_assoc.csv};
\addplot+[thick,mark=none,y filter/.expression={y/2.1},green] table[x=size,y=mask_12,col sep=comma] {results_csv/ait_assoc.csv};
\addplot+[thick,mark=none,y filter/.expression={y/2.1},blue,dotted] table[x=size,y=mask_19,col sep=comma] {results_csv/ait_assoc.csv};
\addplot+[thick,mark=none,y filter/.expression={y/2.1},gray,dotted] table[x=size,y=mask_20,col sep=comma] {results_csv/ait_assoc.csv};
\addplot+[thick,mark=none,y filter/.expression={y/2.1},red,dotted] table[x=size,y=mask_12-13,col sep=comma] {results_csv/ait_assoc.csv};
\addplot+[thick,mark=none,y filter/.expression={y/2.1},red] table[x=size,y=mask_12-19,col sep=comma] {results_csv/ait_assoc.csv};
\draw[gray,densely dotted,thick] (axis cs:65536,100) -- (axis cs:65536,500);
\node[draw=none, text width=2cm] at (axis cs:300000,165) {Size of one set};
\draw[->,black,thick] (axis cs:32768,200) -- (axis cs:65536,270);
\draw[gray,densely dotted,thick] (axis cs:16777216,100) -- (axis cs:16777216,500);
\node[draw=none, text width=2cm] at (axis cs:1600000,460) {Size of AIT};
\draw[->,black,thick] (axis cs:6000000,460) -- (axis cs:16777216,420);

\legend{No mask, mask bit 12, mask bit 19, mask bit 20, mask bit 12-13, mask bit 12-19}
\end{axis}
\end{tikzpicture}
}
\end{subfigure}
\begin{subfigure}[b]{0.45\hsize}
\adjustbox{}{% \vspace{-0.4cm}
\begin{tikzpicture}
\begin{axis}[
mlineplot,
style={font=\footnotesize},
xlabel={\# AIT cache lines ($N$)},
ylabel={AIT miss rate},
% title={(b) Reverse order},
% ylabel style={text width=2cm,align=center},
width=1\hsize,
scaled y ticks=false,
xtick pos=bottom,
ytick pos=left,
xmin=0,
xmax=32,
xtick={0,8,16,24,32},
ymin=0,
ymax=1,
height=3.5cm,
legend columns=1,
legend cell align=left,
legend image post style={scale=0.6}, 
legend style={nodes={scale=0.75},draw=none,anchor=north,at={(0.5,1.6)}},
]
\addplot+[thick,mark=triangle,mark size=1,draw=blue,mark indices={2,4,6,...,32}] table[x=Iteration,y=Rate,col sep=comma] {results_csv/ait_replacement_same_order.csv};
\addplot+[thick,mark=*,mark size=1,draw=red!60,mark indices={2,4,6,...,32}] table[x=Iteration,y=Rate,col sep=comma] {results_csv/ait_replacement_reverse_order.csv};
\addplot+[thick,mark=o,mark size=1,draw=black!20,mark indices={2,4,6,...,32}] table[x=Iteration,y=Rate,col sep=comma] {results_csv/ait_replacement_random_order.csv};
\legend{Same order, Reverse order, Random order}
\end{axis}
\end{tikzpicture}
}
% \vspace{-0.2cm}
\end{subfigure}
\label{fig:ait_reverse}
\vspace{-0.4cm}
\end{figure}

\subsection{Wear Levelling}
Here we reproduce the results showing wear-levelling behaviour in Optane.
\corresponds{figure 8, 9}
\begin{figure}[H]
% \centering
\adjustbox{max width=\hsize}{
\begin{subfigure}[t]{0.58\hsize}
\adjustbox{}{\begin{tikzpicture}
\begin{axis}[
mlineplot,
style={font=\footnotesize},
xlabel={Write count ($\times 10^3$)},
ylabel={Write latency (\textmu s)},
% title={(b) Reverse order},
% ylabel style={text width=2.8cm,align=center},
width=1\hsize,
scaled y ticks=false,
xtick pos=bottom,
ytick pos=left,
xmin=0,
xmax=200,
ymin=0,
ymax=60,
height=3.3cm,
]
\addplot+[
    thick,mark=none, 
    y filter/.expression={y/2100}, 
    x filter/.expression={x/1000-20} % skip initialization
] table[x=Iteration,y=Latency,col sep=comma] {results_csv/wearlevel_write.csv};
\end{axis}
\end{tikzpicture}
\vspace{-0.4cm}
}
\end{subfigure}
\begin{subfigure}[t]{0.34\hsize}
\adjustbox{}{\begin{tikzpicture}
\begin{axis}[
ybar,
style={font=\footnotesize},
xlabel={Wear-leveling latency (\textmu s)},
ylabel={Frequency [\%]},
% ylabel={Density},
% title={(a)},
% ylabel style={text width=2.5cm,align=center},
width=1\hsize,
scaled y ticks=false,
xtick pos=bottom,
ytick pos=left,
ytick = {0,0.1,0.2,0.3,0.4},
yticklabels={0\,\%,10\,\%,20\,\%,30\,\%,40\,\%},
xmin=46,
xmax=52,
ymin=0,
ymax=0.4,
height=3.2cm,
legend columns=1,
legend cell align=left,
legend image post style={scale=0.6}, 
legend style={nodes={scale=0.75},draw=none},
]
\addplot+[   
    hist={bins=200,data min=0, data max=122928},
    y filter/.expression={y==0 ? -1 : y/100},
    x filter/.expression={x/2100},
    draw=blue,
    fill=blue!50,
    opacity=0.7
] table[y index=0] {results_csv/wearlevel_write_lat_dist.csv};
\end{axis}
\end{tikzpicture}
\vspace{-0.4cm}
}
\end{subfigure}}
\label{fig:wear_leveling_timing}
\vspace{-0.4cm}
\end{figure}
\begin{figure}[H]
\centering
\begin{tikzpicture}
\begin{axis}[
ybar,
style={font=\footnotesize},
xlabel={Number of writes ($\times 10^3$)},
ylabel={Frequency [\%]},
% ylabel={Density},
% title={(b)},
% ylabel style={text width=2.5cm,align=center},
width=1\hsize,
scaled y ticks=false,
xtick pos=bottom,
ytick pos=left,
ytick = {0,0.2,0.4,0.6,0.8},
yticklabels={0\,\%,20\,\%,40\,\%,60\,\%,80\,\%},
xmin=1,
xmax=12,
ymin=0,
ymax=0.8,
height=3.3cm,
legend columns=2,
legend cell align=left,
legend image post style={scale=0.6}, 
legend style={nodes={scale=0.75},draw=none,anchor=north,at={(0.5,0.95)}},
]
\addplot+[    
    hist={bins=600,data min=0, data max=63172},
    y filter/.expression={y==0 ? -1 : y/100},
    x filter/.expression={x/1000},
    draw=blue,
    fill=blue!50,
    opacity=0.7,
    pattern=north east lines
] table[y index=0] {results_csv/wearlevel_write_gap_dist.csv};
\addplot+[
    hist={bins=600,data min=0, data max=63172},
    y filter/.expression={y==0 ? -1 : y/100},
    x filter/.expression={x/1000},
    draw=red,
    fill=red!70,
    opacity=0.7
] table[y index=0] {results_csv/wearlevel_writeread_gap_dist.csv};
\legend{Write-flush only, Write-flush-read}
\end{axis}
\end{tikzpicture}
\vspace{-0.5cm}

\label{fig:wear-leveling-readwrite}
\vspace{-0.4cm}
\end{figure}

Here, we reproduce results that depict the wear-levelling and counter granularities.
\corresponds{figure 10}
\begin{figure}[H]
% \vspace{-0.9cm}
\centering
\begin{subfigure}[b]{0.49\hsize}
\begin{tikzpicture}
\begin{axis}[
mlineplot,
style={font=\footnotesize},
xlabel={Address distance: $\mathit{offset}\cdot 256$ (Bytes)},
xlabel style={text width=3.5cm,align=center},
ylabel={\# Extra writes to trigger wear-leveling},
% title={(a) },
ylabel style={text width=2.35cm,align=center},
width=0.92\hsize,
scaled y ticks=false,
xtick pos=bottom,
ytick pos=left,
xmin=0,
xmax=4096,
xtick={0,256,1024,2048,4096},
xticklabels={0,256,1 kB,2 kB,4 kB},
ymin=0,
ymax=15000,
height=3.5cm,
yticklabel style={/pgf/number format/1000 sep=},
xticklabel style={rotate=60,/pgf/number format/1000 sep=},
]
\addplot+[thick,mark=*,mark size=1] table[x=offset,y=remaining_writes,col sep=comma] {results_csv/wearlevel_counter.csv};
\draw[gray,densely dotted,thick] (axis cs:256,0) -- (axis cs:256,15000);
\draw[black,thick] (axis cs:20,4000) -- (axis cs:250,4000);
\draw[black,thick] (axis cs:20,3400) -- (axis cs:20,4600);
\draw[black,thick] (axis cs:250,3400) -- (axis cs:250,4600);
\node[draw=none, text width=1.4cm] at (axis cs:2600,5000) {Counter Granularity};
\draw[->,black,thick] (axis cs:1000,5000) -- (axis cs:400,4000) ;
\end{axis}
\end{tikzpicture}
\vspace{-0.4cm}

\end{subfigure}
\begin{subfigure}[b]{0.49\hsize}
\begin{tikzpicture}
\begin{axis}[
mlineplot,
style={font=\footnotesize},
xlabel={Address distance: $\mathit{offset}\cdot 256$ (Bytes)},
xlabel style={text width=3.5cm,align=center},
ylabel={\# Extra writes to trigger wear-leveling},
% title={(b)},
ylabel style={text width=2.35cm,align=center},
width=0.92\hsize,
scaled y ticks=false,
xtick pos=bottom,
ytick pos=left,
xmin=0,
xmax=8192,
xtick={0,2048,4096,6144,8192},
xticklabels={0,2 kB, 4 kB, 6 kB, 8 kB},
ymin=0,
ymax=15000,
height=3.5cm,
yticklabel style={/pgf/number format/1000 sep=},
xticklabel style={rotate=60,/pgf/number format/1000 sep=},
]
\addplot+[thick,mark=*,mark size=1] table[x=offset,y=writes_to_wl,col sep=comma] {results_csv/wearlevel_gran.csv};
\draw[gray,densely dotted,thick] (axis cs:4096,0) -- (axis cs:4096,15000);
\draw[black,thick] (axis cs:60,4000) -- (axis cs:4000,4000);
\draw[black,thick] (axis cs:60,4600) -- (axis cs:60,3400);
\draw[black,thick] (axis cs:4000,4600) -- (axis cs:4000,3500);
\node[draw=none, text width=2.4cm] at (axis cs:4500,2000) {Remap Granularity};
% \draw[->,black,thick] (axis cs:3200,13000) -- (axis cs:2000,12500) ;
\end{axis}
\end{tikzpicture}
\vspace{-0.4cm}

\end{subfigure}
\label{fig:wear-leveling-granularity}
\end{figure}


\subsection{Read-Write Contention}
Here, we show how read performance in Optane degrades when there are concurrent writes.
\corresponds{figure 11}
\begin{figure}[H]
\centering
\begin{tikzpicture}
\begin{axis}[
% mlineplot,
ybar,
style={font=\footnotesize},
xlabel={Read latency (ns)},
ylabel={Frequency [\%]},
% ylabel={Density},
% title={(b) Reverse order},
% ylabel style={text width=4cm,align=center},
width=1\hsize,
scaled y ticks=false,
xtick pos=bottom,
ytick pos=left,
ytick = {0,0.2,0.4,0.6,0.8},
yticklabels={0\,\%,20\,\%,40\,\%,60\,\%,80\,\%},
xmin=300,
xmax=1200,
% xtick={30,...,128},
ymin=0,
ymax=0.9,
xticklabel style={/pgf/number format/1000 sep=},
height=3.3cm,
legend columns=2,
legend cell align=left,
legend image post style={scale=0.6}, 
legend style={nodes={scale=0.75},draw=none},
]
\addplot+[    
    hist={bins=200,data min=0, data max=2389},
    x filter/.expression={x/2.1},
    y filter/.expression={y==0 ? -10 : y/100},
    draw=blue,
    fill=blue!50,
    opacity=0.7
] table[y index=0] {results_csv/rw_cont_read_read.csv};
\addplot+[    
    hist={bins=200,data min=0, data max=2389},
    x filter/.expression={x/2.1},
    y filter/.expression={y==0 ? -10 : y/100},
    draw=red,
    fill=red!10,
    opacity=0.7,
    pattern=north east lines
] table[y index=0] {results_csv/rw_cont_read_write_10.csv};
\addplot+[    
    hist={bins=200,data min=0, data max=2389},
    x filter/.expression={x/2.1},
    y filter/.expression={y==0 ? -10 : y/100},
    draw=red,
    fill=red!30,
    opacity=0.7,
] table[y index=0] {results_csv/rw_cont_read_write_30.csv};
\addplot+[    
    hist={bins=200,data min=0, data max=2389},
    x filter/.expression={x/2.1},
    y filter/.expression={y==0 ? -10 : y/100},
    draw=red,
    fill=red!50,
    opacity=0.7,
    pattern=horizontal lines
] table[y index=0] {results_csv/rw_cont_read_write_50.csv};
\addplot+[    
    hist={bins=200,data min=0, data max=2389},
    x filter/.expression={x/2.1},
    y filter/.expression={y==0 ? -10 : y/100},
    draw=red,
    fill=red!100,
    opacity=0.7,
] table[y index=0] {results_csv/rw_cont_read_write_100.csv};
\addplot+[    
    hist={bins=200,data min=0, data max=2389},
    x filter/.expression={x/2.1},
    y filter/.expression={y==0 ? -10 : y/100},
    draw=black,
    fill=black!50,
    opacity=0.7,
    pattern=north west lines
] table[y index=0] {results_csv/rw_cont_read.csv};
\legend{Co-located reader (100\,\%), Co-located writer (10\,\%), Co-located writer (30\,\%), Co-located writer (50\,\%), Co-located writer (100\,\%),No co-located reader/writer, }
\end{axis}
\end{tikzpicture}
\vspace{-0.5cm}

\label{fig:read_write_contention}
\vspace{-0.2cm}
\end{figure}

\section{Local Covert Channel}
The following table contains results for the local covert channel experiment.
\newcommand{\covertdata}[1]{\accessdatfile{covert}{#1}}
\begin{table}[H]
    \label{tab:local_covert}
	\setlength{\tabcolsep}{3pt}
    \centering
    \small
    \begin{tabular}{l|r|r|r|r}
        \toprule
        Channel &  \multicolumn{1}{c}{BW ($kbit/s$)} & \multicolumn{1}{c}{Acc (\%)} & \multicolumn{1}{c}{$\sigma_{BW}$} & \multicolumn{1}{c}{$\sigma_{Acc}$}\\
        \toprule
        RMW & $\covertdata{rmw_kbps_mean}$  & $\covertdata{rmw_acc_mean}$  & $\covertdata{rmw_kbps_stdev}$ & $\covertdata{rmw_acc_stdev}$ \\
        AIT & $\covertdata{ait_kbps_mean}$  & $\covertdata{ait_acc_mean}$  & $\covertdata{ait_kbps_stdev}$ & $\covertdata{ait_acc_stdev}$ \\
        Contention & $\covertdata{cont_kbps_mean}$  & $\covertdata{cont_acc_mean}$  & $\covertdata{cont_kbps_stdev}$ & $\covertdata{cont_acc_stdev}$ \\
        % \midrule
        \bottomrule
    \end{tabular}
    \vspace{-0.2cm}
\end{table}


\section{Keystroke Attack}
The following figure depicts how our side-channel can detect keystrokes. Moreover, we also present accuracy numbers.
\corresponds{figure 15}
\begin{figure}[H]
\adjustbox{max width=\hsize}{\begin{tikzpicture}
    \begin{groupplot}[
    group style={group size= 1 by 2, vertical sep=0.5cm},
    ]
    
    \nextgroupplot[style={font=\footnotesize},
    ylabel={$\Delta T$ in ms},
    % y label style={align=center,text width=2cm},
    xtick pos=bottom,
    ytick pos=left,
    % xticklabel style={/pgf/number format/1000 sep=},
    xticklabel=\empty,
    yticklabel style={/pgf/number format/1000 sep=},
    width=\hsize,
    legend image post style={scale=0.6}, 
    %axis y line*=left,
    height=3cm,
    legend style={at={(1.0,1.0)}, anchor=north east, legend columns=2, font=\footnotesize,draw=none,fill=none},]
    \addplot[blue!80] table [x=index,y=reference, col sep=comma] {results_csv/keystroke_attack_result_time.csv};
    \addlegendentry{Reference};
    \nextgroupplot[style={font=\footnotesize},
    yshift=0.35cm,
    ylabel={$\Delta T$ in ms},
    xlabel={Keystroke index},
    % y label style={align=center,text width=2cm},
    xtick pos=bottom,
    ytick pos=left,
    xticklabel style={/pgf/number format/1000 sep=},
    yticklabel style={/pgf/number format/1000 sep=},
    width=\hsize,
    legend image post style={scale=0.6}, 
    %axis y line*=left,
    height=3cm,
    legend style={at={(1.0,1.0)}, anchor=north east, legend columns=2, font=\footnotesize,draw=none,fill=none},]
    \addplot[red!40] table [x=index,y=result, col sep=comma] {results_csv/keystroke_attack_result_time.csv};
    \addlegendentry{RMW side-channel};
    \end{groupplot}
\end{tikzpicture}
\vspace{-0.5cm}
}
\label{fig:keystroke_attack_results_time}
\vspace{-0.4cm}
\end{figure}

\newcommand{\keystrokedata}[1]{\accessdatfile{keystroke}{#1}}
\begin{itemize}
    \item Precision: $\keystrokedata{prec_mean} \pm \keystrokedata{prec_stdev} \%$
    \item Recall: $\keystrokedata{recall_mean} \pm \keystrokedata{recall_stdev} \%$
\end{itemize}

\section{Remote Covert Channel}
Here, we show results after running the remote covert channel experiment.

\newcommand{\remotecovertdata}[1]{\accessdatfile{remote-covert}{#1}}
\begin{itemize}
    \item Bandwidth: $\remotecovertdata{bws_mean} \pm \remotecovertdata{bws_stdev} \% $
    \item Accuracy: $\remotecovertdata{acc_mean} \pm \remotecovertdata{acc_stdev} \% $
    \item Packets per bit: $\remotecovertdata{packets_per_bit_mean} \pm \remotecovertdata{packets_per_bit_stdev} \% $
\end{itemize}


\section{Noteboard Attack}
Here, we show results after running the noteboard attack.
\newcommand{\noteboarddata}[1]{\accessdatfile{noteboard}{#1}}
\begin{itemize}
    \item Accuracy: $\noteboarddata{noteboard_acc_mean} \pm \noteboarddata{noteboard_acc_stdev} \% $
\end{itemize}
\end{document}  
